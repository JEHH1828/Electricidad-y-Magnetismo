\documentclass[10pt]{article}
\usepackage{parskip}
\usepackage[utf8]{inputenc}
\usepackage[left=2.00cm, right=2.00cm, top=2.00cm, bottom=2.00cm]{geometry}
\usepackage[spanish]{babel}
\usepackage{graphicx,subfig}
\usepackage{fancyhdr}
\graphicspath{{Imagenes/}}
\usepackage{enumerate} 
\usepackage{multicol}
\begin{document}


\pagestyle{fancy}
\cfoot{}


%Cabeceras
\rhead{Multímetro.}
\lhead{}

%Portada
\begin{titlepage}
	\newgeometry{
		left=25mm,
		right=25mm,
		top=5mm,
		bottom=30mm,
		headheight = 0 mm
	}

	\begin{figure}[t]
		\subfloat{\includegraphics[width=0.15\textwidth]{Logo_IPN}}
		\hspace{0.6\textwidth}
		\subfloat{\includegraphics[width=0.22\textwidth]{LogoEsime}}
	\end{figure}

	\centering
	{\bfseries\Huge Instituto Politécnico Nacional. \par}
	\vspace{1cm}
	{\scshape\Large Ingeniería en Comunicaciones y Electrónica. \par}
	\vspace{0.3cm}
	{\scshape\Large Laboratorio de Electricidad y Magnetismo.  \par}
	\vspace{1cm}
	{\scshape\Huge El Universo de las Mediciones Eléctricas. \par}
	\vspace{1cm}
	{\itshape\Large Multímetro. \par}
	{\Large 2CM13\par}
	\vfill
	{\Large Autores: \par}
	{\Large Daniela Elizabeth Pérez Vargas. \par}
	{\Large Jesús Martinez Amac. \par}
	{\Large José Emilio Hernández Huerta. \par}
	{\Large Nataly Bejarano Garduño.\par}
	{\Large Uriel Grimaldi Díaz.  \par}
	\vfill
	{\Large Mayo 2023. \par}

\end{titlepage}

\tableofcontents
\newpage

\section{Resumen.}
En la presente práctica, se desarrolla el uso del multímetro con enfásis en las funciones de medición de resistencia, continuidad, vóltmetro (Corriente Directa y Corriente Alterna) y Amperímetro. 

\begin{multicols}{2}

\section{Objetivo.}

El alumno será capaz de describir las características y funcionamiento del multímetro así como manejar correctamente dicho instrumento para realizar mediciones de las 3 magnitudes eléctricas fundamentales (Resistencia, Potencial eléctrico y Corriente eléctrica).

\section{Introducción.}

La herramienta fundamental del técnico o ingeniero especializado en la manipulación de componentes electrónicos es el multímetro, dicha herramienta permite al profesional obtener mediciones de magnitudes eléctricas que le puedan resultar útiles,
estas magnitudes suelen ser , la resistencia eléctrica, el potencial eléctrico, la corriente eléctrica,frecuencia, inductancia, capacitancia además de otras utilidades como la continuidad y la prueba de diodos.

\section{Marco teórico.}



\section{Descripción de materiales.}

\begin{center}

	\includegraphics[scale = 0.1]{Imagenes/Material/MultiD.jpeg}\\
	Multímetro Digital.

	\includegraphics[scale = 0.1]{Imagenes/Material/MultiA.jpeg}\\
	Multímetro Analógico.

	\includegraphics[scale = 0.1]{Imagenes/Material/Puntas.jpeg}\\
	Puntas para multímetro.

	\includegraphics[scale = 0.1]{Imagenes/Material/PilaD.jpeg}\\
	Pila tipo "D" de 1.5 V.

	\includegraphics[scale = 0.1]{Imagenes/Material/PilaDes.jpeg}\\
	Pila de valor desconocido.

	\includegraphics[scale = 0.1]{Imagenes/Material/Protoboard y jumper.jpeg}\\
	Protoboard y Jumpers.

	\includegraphics[scale = 0.1]{Imagenes/Material/Caimanban.jpeg}\\
	Cables <<Caimán-Banana>>.

	\includegraphics[scale = 0.1]{Imagenes/Material/banban.jpeg}\\
	Cables <<Banana-Banana>>.

	\includegraphics[scale = 0.1]{Imagenes/Material/FuenteVariable.jpeg}\\
	Fuente de alimentación regulada.

	\includegraphics[scale = 1]{Imagenes/Material/Resistencias.jpeg}\\
	Resistencias de valores entre 1K$\Omega$ y 9K$\Omega$

\end{center}


\section{Desarrollo experimental.}

\subsection{Reconocimiento del multímetro.}

Tome el multímetro digital y reconozca lo siguiente:
\begin{enumerate}
	\item La marca y modelo del multímetro.
	\item Como encender el multímetro.
	\item Cuántas posiciones y cuáles son los rangos del multímetro para medir voltajes de Corriente Directa.
	\item Cuántas posiciones y cuáles son los rangos del multímetro para medir corrientes de Corriente Directa y Corriente Alterna.
	\item Cuántos y cuáles son los rangos del multímetro para medir resistencia.
	\item Otras funciones,interruptores y selectores.
\end{enumerate}

\subsection{Mediciones de resistencia(óhmetro).}

\begin{enumerate}
	\item Encienda el multimetro y coloque la perilla en Ohms.
	\item Anote los valores utilizando el código de colores para resistores.
	\item Mida los resistores y anote los valores obtenidos en una tabla.
	\item Compare el valor nominal con el valor medido.
\end{enumerate}

\subsection{Mediciones de continuidad.}

\begin{enumerate}
	\item Utilizando el medidor de continuidad o en la escala de resistencia más baja identifique como está constituida una protoboard.
	\item Elabore un diagrama según lo observado.
\end{enumerate}

\subsection{Mediciones de voltaje(vóltmetro).}

\begin{enumerate}
	\item Coloque la perilla del multímero en volts y ubique correctamente el selector de tipo de corriente en Corriente Directa.
	\item Mida el voltaje de las pilas y anote sus valores.
	\item Utilice la fuente regulada, ubique las salidas de corriente directa y realice tres mediciones diferentes respetando la polaridad.
\end{enumerate}

\subsection{Mediciones de voltaje de corriente alterna(vóltmetro).}

\begin{enumerate}
	\item Ubique un contacto como el mostrado en la siguiente imagen.\\
	\begin{center}
		\includegraphics{Imagenes/Fotos/Contacto.png}
	\end{center}
	\item Mida el voltaje entre salidas y anote su valor.
	\item Identifique en el contacto cuál es la fase,el neutro y la tierra física.
\end{enumerate}

\subsection{Mediciones de intensidad de corriente continua(Amperímetro.)}
\begin{enumerate}
	\item Arme el siguiente circuito.\\
	\begin{center}
		\includegraphics{Imagenes/Fotos/Circuito.png}
	\end{center}
	\item Seleccione una escala de medición adecuada y coloque el selector de Corriente Alterna y Corriente Directa en Corriente Directa.
	\item Mida la corriente en el circuito.
	\item Calcule el valor teórico de la corriente del circuito y compare con el valor medido.
\end{enumerate}

\section{Discusión de materiales .}


\section{Análisis y resultados.}

\subsection{Reconocimiento del multímetro.}

\subsection{Mediciones de resistencia(óhmetro).}

\subsection{Mediciones de continuidad.}

\subsection{Mediciones de voltaje(vóltmetro).}

\subsection{Mediciones de voltaje de corriente alterna(vóltmetro).}

\subsection{Mediciones de intensidad de corriente continua(Amperímetro.)}




\section{Conclusiones.}

\subsection*{Daniela Elizabeth Pérez Vargas.}

\subsection*{Jesús Martinez Amac.}
\subsection*{José Emilio Hernández Huerta.}
\subsection*{Nataly Bejarano Garduño.}
\subsection*{Uriel Grimaldi Díaz.}

\end{multicols}
\newpage
\clearpage
\begin{thebibliography}{0}
	\bibitem{citekey}
\end{thebibliography}

\end{document}
