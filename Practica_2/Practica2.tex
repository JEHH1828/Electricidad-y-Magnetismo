\documentclass[spanish,10pt,a4paper,onecolumn]{article}

\usepackage[utf8]{inputenc}
\usepackage[left=2.00cm, right=2.00cm, top=2.00cm, bottom=2.00cm]{geometry}
\usepackage[spanish]{babel}
\usepackage{graphicx,subfig}
\usepackage{fancyhdr}
\graphicspath{{Imagenes/}}
\usepackage{lipsum}
\usepackage{enumerate} 
\usepackage{multicol}
\begin{document}


\pagestyle{fancy}
\cfoot{}


%Cabeceras
\rhead{Distribución de las cargas eléctricas}
\lhead{}

%Portada
\begin{titlepage}
	\newgeometry{
		left=25mm,
		right=25mm,
		top=5mm,
		bottom=30mm,
		headheight = 0 mm
	}

	\begin{figure}[t]
		\subfloat{\includegraphics[width=0.15\textwidth]{Logo_IPN}}
		\hspace{0.6\textwidth}
		\subfloat{\includegraphics[width=0.22\textwidth]{LogoEsime}}
	\end{figure}

	\centering
	{\bfseries\Huge Instituto Politécnico Nacional. \par}
	\vspace{1cm}
	{\scshape\Large Ingeniería en Comunicaciones y Electrónica. \par}
	\vspace{0.3cm}
	{\scshape\Large Laboratorio de Electricidad y Magnetismo.  \par}
	\vspace{1cm}
	{\scshape\Huge ECHALE TIERRA \par}
	\vspace{1cm}
	{\itshape\Large Distribución de las cargas eléctricas en los conductores. \par}
	{\Large 2CM13\par}
	\vfill
	{\Large Autores: \par}
	{\Large Daniela Elizabeth Pérez Vargas. \par}
	{\Large Jesús Martinez Amac. \par}
	{\Large José Emilio Hernández Huerta. \par}
	{\Large Nataly Bejarano Garduño..\par}
	{\Large Uriel Grimaldi Díaz.  \par}
	\vfill
	{\Large Abril 2023. \par}

\end{titlepage}

\tableofcontents
\newpage




\section{Resumen.}


\begin{multicols}{2}
\section{Objetivos.}
lorem ipsum dolor sit amet, consectetur adipiscing elit, sed do eiusmod tempor incididunt ut labore et dolore magna aliqua. Ut enim ad minim veniam, quis nostrud
\subsection{Generales.}

\subsection{Particulares.}



\section{Introducción.}



\section{Marco teórico.}



\section{Descripción de materiales.}



\section{Desarrollo experimental.}

\subsection{El electroscopio.}

\subsection{La experiencia de Cavendish.}

\subsection{Experiencia de Franklin.}

\subsection{Pantalla eléctrica.}

\subsection{Efecto de puntas.}

\subsubsection{Rehilete electrostático.}

\subsubsection{Mechón de cabellos.}

\subsubsection{Experiencia de la vela.}



\section{Análisis y resultados.}



\section{Conclusiones.}

\end{multicols}
\newpage
\clearpage
\begin{thebibliography}{0}
	\bibitem{citekey}
\end{thebibliography}

\end{document}