\documentclass[14pt]{article}
\usepackage[utf8]{inputenc}
\usepackage[T1]{fontenc}
\usepackage[spanish]{babel}
\usepackage{graphicx,subfig}
\graphicspath{{./Fotos/}}


% Geometría
\usepackage{geometry}
\geometry{
	left=25mm,
	right=25mm,
	top=30mm,
	bottom=30mm,
	headheight = 35 mm
} 
\usepackage{fancyhdr}
\pagestyle{fancy}
\cfoot{}
\usepackage{lipsum}

% Texto
\renewcommand*\familydefault{\sfdefault} 


\renewcommand{\theenumi}{\Roman{enumi}}
\begin{document}
	%Cabeceras
	\rhead{La Electrostática}
	\lhead{}
	%Portada
	\begin{titlepage}
		\newgeometry{
			left=25mm,
			right=25mm,
			top=5mm,
			bottom=30mm,
			headheight = 0 mm
		}
		\begin{figure}[t]
			\subfloat{\includegraphics[width=0.22\textwidth]{Logo_IPN}}
			\hspace{0.6\textwidth}
			\subfloat{\includegraphics[width=0.22\textwidth]{LogoEsime}}
		\end{figure}
		\centering
		{\bfseries\Huge Instituto Politécnico Nacional \par}
		\vspace{1cm}
		{\scshape\Large Ingeniería en Comunicaciones y Electrónica \par}
		\vspace{0.3cm}
		{\scshape\Large Lab. de Electricidad y Magnetismo  \par}
		\vspace{1cm}
		{\scshape\Huge ¿Toques? \par}
		\vspace{1cm}
		{\itshape\Large La Electrostática\par}
		{\Large 2CM13\par}
		\vfill
		{\Large Autores: \par}
		{\Large José Emilio Hernández Huerta \par}
		{\Large Nataly \par}
		{\Large Daniela Elizabeth Pérez Vargas \par}
		{\Large Jesús \par}
		{\Large Uriel Grimaldi Díaz  \par}
		\vfill
		{\Large Abril 2023 \par}
	\end{titlepage}
	%Indice
	 \tableofcontents
	 
	\newpage
	%Inicia el reporte
	\section{Resumen}
	\section{Objetivos} 
	\subsection{Generales}
	\subsection{Particulares}
	\subsection{Introducción}
	Los humanos siempre hemos visto como la electricidad de alguna u otra forma ya sea con fenómenos meteorológicos como son los rayos o pequeñas chispas que saltan entre nuestras cobijas, inclusivamente en nuestros cuerpos desde las descargas que produce nuestro cerebro para controlar el cuerpo y las descargas de estática que experimentamos cuando nos cargamos eléctricamente. Y para poder comprender todos estos fenómenos y que es la electrostática tenemos que definir algunos conceptos importantes así como sus diferentes formulas y características. 
	\section {Marco Teórico}
	  	
	\subsection{Definición}
	Empezando con lo más simple, la electrostática es la parte de la física que estudia la electricidad en la materia y los
	fenómenos producidos por cargas eléctricas en reposo.[1]
	La electrostática describe los fenómenos que tienen lugar en sistemas donde distribuciones de carga eléctrica mantienen su localización invariante en el tiempo. En
	otras palabras, los cuerpos cargados deben permanecer en reposo. Aún más, cada porción de carga debe permanecer en reposo dentro del cuerpo cargado.[4]
	\subsection{Carga Eléctrica}
	Desde la antigua Grecia, los filósofos de la época ya conocían la existencia del ámbar y que al frotarlo este atraía trocitos de ámbar
	La carga eléctrica es una magnitud fundamental de la física, responsable de la interacción electromagnética.
	1831
	1879 Se introducen los conceptos de carga eléctrica, fuerza
	electromagnética, campo, corriente, energía potencial electrostática, etc
	James Clerk Maxwell puso las ideas de Faraday en lo que se conoce como
	las ecuaciones de Maxwell
	
	\subsection{Ley de Coulomb}
	Esta ley fue creada por Charles Coulomb (1736-1806) cuando midió las magnitudes de las fuerzas eléctricas entre objetos de carga. Cada carga puntual ejerce una fuerza sobre la otra, la cual esta dirigida a lo largo de la linea entre las cargas ($r.$) y posee igual magnitud. [2]\\
	\begin{math}
		 \vec{F} = k_{e} \frac{q_{1}q_{2}}{r{2^2}} \\  
		 k = 8.99*10^9 \frac{Nm^2}{C^2} \\
		 \epsilon_{0} = 8.55*10^-12 \frac{C^2}{Nm^2} \\
		 k = \frac{1}{4 \pi \epsilon_{0}} \\
	\end{math}
	
		\subsection{Propiedades}
		• La carga eléctrica se conserva \\
		• En un átomo neutro, las cargas 
		positiva y negativa tienen la misma
		magnitud \\
		• La carga esta cuantizada y su
		unidad fundamental es
		$e = 1.6*10^{-19}C$ \\
		• En el sistema SI la unidad de
		carga es el Coulomb \\
	\subsection{Tipos de Materiales}
	En el mundo en que vivimos los materiales tienen diferentes clasificaciones así dependiendo de sus propiedades, y en el este caso en particular hablaremos de su capacidad para conducir o transferir la carga eléctrica clasificándolos en aislantes, conductores y semiconductores 
	\subsubsection{Aislantes}
	Los electrones están ligados a los átomos por lo que la transferencias de carga es nula. Algunos ejemplos son el caucho, la madera, algunos plásticos, etc. 
	\subsubsection{Conductores}
	Los electrones son libres de moverse por el material. Ejemplos de estos son los metales, como el cobre, oro, aluminio, etc.
	\subsubsection{Semiconductores}
	Los semiconductores son un tipo especial de materiales debido a que
	presentan la característica de que se pueden comportar como conductores
	o como aislantes, dependiendo de las condiciones en que se utilicen. Por ejemplo el Silicio, Germanio, Azufre, Indio, etc. 
	\newpage
	\begin{thebibliography}{0}
		\bibitem{HHS}Grupo Educacional Heinrich:
		New Heinrich High School
		\bibitem{Carga}2017 Departamento de Física
		Universidad de Sonora
		\bibitem{no me acuerdo}Electrostatica Universidad Nacional de Rosario 
	\end{thebibliography}
\end{document}